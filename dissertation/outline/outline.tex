\documentclass[12pt,pdftex]{article}
\usepackage{doublespace}

\topmargin -2.0cm
\oddsidemargin 0in
\evensidemargin 0in
%\footheight 0.0pt
\headheight 2\baselineskip
\textwidth 6.0in
\textheight 9.0in



\begin{document}
\begin{spacing}{1.5}

  
\section{Introduction}
I have done little to the introduction beyond what you included in an
email you sent me a while back.  The major goal is to motivate the
work by describing some of the difficulties that arise in CFD:
\begin{itemize}
  \item monotonicity
  \item oscillations
  \item stabilization
  \item accuracy
  \item reliability
\end{itemize}

The importance of these properties is described.  The contributions of
the work are listed as:
\begin{enumerate}
  \item[] \textbf{Major Contributions:}
    
  \item monotonicity properties for convection-reaction-diffusion
  discretizations
  
  \item nested grid approaches
    
  \item subgrid strategies for solution enhancement and stabilization
    
  \item applications and phenomenological studies

  \item[] \textbf{Other Contributions:}
    
  \item adaptive refinement research
    
  \item parallel data structures and related issues
\end{enumerate}


\section{Monotonicity}
\subsection{Limiter approaches}
Describe the 1D and 2D flux-limiter approach devised by MacKinnon and
Carey.  Emphasize the extension to 2D and the resulting numerical
experiments.  (Material in this chapter will be drawn heavily from the
in-progress 2D paper.)  We are working on the 2D paper right now, and
I want to re-format the results.  Right now it is information-overload.

\subsection{Multiplier approaches}
(Is this worth exploring in more detail?) Describe the theory and
numerical experiments from the SIAM Geosciences work when a Lagrange
multiplier is used to help enforce positivity.  Incorporate some of
Wolfgang's improvements into the existing framework.

\section{Nested Grids}
\subsection{Uniform meshes}
Brief overview of CMG theory and examples from the Lipnikov paper.
Repeat those examples in 2D and 3D with other linear solvers used as
the smoother.

\subsection{Adaptive meshes}
Applications of non-uniform nested refinement to steady applications.
The approach here is to take a steady-state (or time-step to steady
state) on a coarse mesh, uniformly refine, selectively coarsen,
repeat...  I have done preliminary work here in 2D with a low Rayleigh
number natural convection test case with promising results.  More work
is needed to help determine optimal coarsening fractions, etc...  This
approach could easily be applied to some of the large-scale 3D
problems in the RBM application.


\section{Subgrid Strategies}
The work here is still in the early stages.  The framework is there
for windowing regions of a problem and either re-solving the entire
problem with a refined window or just solving on the refined window
with polluted boundary data.

Initial results suggest that the coarse-fine interface has significant
error (even in the case of re-solving the entire problem) but the
features (gradients, etc...) of the windowed solution are
significantly better.  At this point I have only applied this to a
simple Poisson problem.  The next test case will be a
convection-dominated problem with a strong boundary layer.

\section{Adaptive Refinement}
(I am not sure about the positioning of this section.  Maybe it should
come after the monotonicity work \& precede the Nested Grids stuff
since that uses adaptive meshes.)

This chapter describes (briefly) the benefits of adaptive meshes.  It
then describes the general approach:  coarse solution, error
estimation, mesh refinement, solution projection/restriction, repeat.
The approach for transient applications will also be discussed.

Difficulties such as hanging nodes and hybrid 3D meshes will be
described.  libMesh will briefly be described as a novel contribution.

\section{Parallel Issues \& Data Structures}
The need for parallel computers will be briefly described since it is
widely accepted.  The emphasis will be on the difficulties encountered
when the adaptive approaches used in the present work are applied to
parallel machines and new approaches to handling these difficulties.
The parallel data structures that are used to enable efficient AMR on
parallel machines will be discussed.

\section{Applications \& Phenomenological Studies}

\subsection{AMR for the Elder Problem}
Extend the work that went into the Japan paper. We had some open
questions about the quality of the adaptive mesh.  I do think it will
be worth presenting the Elder problem due to its unique sensitivity
and necessity of proper mesh resolution.

\subsection{Incompressible Flow \& Transport}
I have done a lot of work on natural convection in various
geometries.  There is also some RBM work that I have done along with
John.  I am not sure what (if anything) from this should be presented.

\subsection{Compressible Flows}
Perhaps discuss the application area with the committee?


\section{Conclusions}
  
\end{spacing}
\end{document}
