\documentclass[10pt]{report}

\usepackage[T1]{fontenc}
\usepackage{ebgaramond}
%\usepackage{times}
\usepackage{layout}
\usepackage{multibib}
\usepackage[ManyBibs,openbib]{currvita}
\usepackage[
pdftex,
backref,
bookmarks=true,
bookmarksopen=true,
colorlinks=true,
linkcolor=black,
citecolor=black,
filecolor=blue,
urlcolor=black,
plainpages=false,
pdfpagelabels,
pdftitle={Curriculum Vitae},
pdfauthor={Benjamin Kirk},
pdfsubject={Vitae}
]
{hyperref}

\setlength{\textwidth}{6.5in}
\setlength{\textheight}{9in}
\setlength{\marginparwidth}{0pt}
\setlength{\oddsidemargin}{0in}
\setlength{\evensidemargin}{0in}
\setlength{\topmargin}{0pt}
\setlength{\headheight}{-20pt}
\setlength{\headsep}{0pt}

\newcites{journal}{journal}
\newcites{conf}{conf}
\newcites{proc}{proc}
\newcites{rep}{rep}
\newcites{shortcourses}{shortcourses}


\begin{document}

%\layout

\begin{cv}{\centerline{\Large Benjamin S. Kirk}\\
    \centerline{\large Curriculum Vitae}}

  \vspace{-0.5em}
  \hskip-1.5em
  \begin{minipage}[h]{.5\textwidth}
    %Born January 26, 1978 \\
    \href{mailto:benkirk@ucar.edu}{\texttt{benkirk@ucar.edu}} \\
    \href{mailto:benjamin.s.kirk@gmail.com}{\texttt{benjamin.s.kirk@gmail.com}} \\
    \url{https://www.linkedin.com/in/benjamin-kirk-87784a14} \\
  \end{minipage}
  \hskip .275\textwidth
  %% \begin{minipage}[h]{.25\textwidth}
  %%   \setlength{\parindent}{0pt}
  %%   2101 NASA Parkway \\
  %%   Mail Code: EG3 \\
  %%   Houston, TX 77058 \\
  %%   832.221.7467 \\
  %% \end{minipage}
  \begin{minipage}[h]{.25\textwidth}
    \setlength{\parindent}{0pt}
    1532 Torrance Drive \\
    League City, TX 77573 \\
    720.378.2744 \\
  \end{minipage}

% Education Section
  \begin{cvlist}{Education}
    \currentpdfbookmark{Education}{education}
    \item[6/2002 -- 5/2007]
      Ph.D.\ Aerospace Engineering,
      The University of Texas at Austin. GPA: 3.94

      Thesis: \href{http://cfdlab.ae.utexas.edu/~benkirk/dissertation.pdf}{\emph{Adaptive Finite Element Simulation of Flow and Transport Applications on Parallel Computers}}, May 2007.
    \item[8/2000 -- 5/2002]
      Masters of Science, Computational and Applied Mathematics,
      The University of Texas at Austin.  GPA: 3.93
    \item[8/1996 -- 5/2000]
      Bachelors of Science, Aerospace Engineering,
      The University of Texas at Austin.  GPA: 4.0
  \end{cvlist}

  \vspace{-.5em}
% Work Experience
  \begin{cvlist}{Work Experience}
    \currentpdfbookmark{Work Experience}{work}
  \item[10/2015 -- 2/2022]
    Chief, Applied Aerosciences \& Computational Fluid Dynamics Branch,  NASA Lyndon B. Johnson Space Center
  \item[8/2018-2/2019]
    Acting Chief, Aeroscience \& Flight Mechanics Division (\emph{rotational assignment})
  \item[6/2013 -- 10/2015]
    Deputy Chief, Applied Aerosciences \& Computational Fluid Dynamics Branch
    \begin{itemize}
    \item
      Technical Authority \& Stewardship of the Aerosicence Disciplines -- Aerodynamics, Aerothermodynamics, Rarefied Gas Dynamics, and Parachute/Decelerator Systems -- for application within NASA's Human Spaceflight Programs.
    \item
      Led the Applied Aerosciences \& Computational Fluid Dynamics Branch through multiple programmatic milestones important to NASA's Human Spaceflight Programs, including development \& certification of multiple flight vehicles through direct support of multiple flight tests with in the Orion and Commercial Crew Programs, as well as sustained support for the International Space Station.

      Notable Milestones Achieved:
      \begin{itemize}
      \item Orion Program's Exploration Flight Test, Ascent Abort Flight Test, Artemis~1 flight readiness.
      \item Commercial Crew Program's development and qualification activities, from inception through SpaceX Crew~3 flight readiness.
      \item International Space Station operations \& certification of multiple new `visiting vehicles` and associated plume-induced aerothermodynamic environment characterization.
      \end{itemize}

    \item
      Strategic planning \& execution of technical work content across a combined government \& contractor workforce of $\approx$35 individuals, with direct responsibility for $\approx$\$10M annual full-cost operating budget.

    \item
      Principal responsibilities for employee recruitment, work assignments, and daily management activities.
    \end{itemize}


  \item[12/2003 -- 2/2022]
    High Performance Computing (HPC) Chief Lab Architect \& System Administrator
    \begin{itemize}
      \item
        Established the governance \& operating model for NASA JSC's Flight  Sciences Lab (FSL), including the FSL User Forum (FUF), with the primary goals of:
        \begin{itemize}
        \item
          Leveraging background as both an HPC system user and administrator to `translate' between these two communities, with focus on explaining the technical `why` and offering `how` in a user-centric approach. The overarching goal of this thrust was to promote mutual understanding of technical needs and challenges, foster mutual respect, and increase overall system performance and user satisfaction.
        \item
          Bridging the knowledge gap between System Users and System Administrators, through focused technology training, code profiling and optimization, workflow process enhancements, and technology deep-dives (e.g. MPI, Lmod modules, Luste, NFS, ZFS \& NFS).
        \item
          Optimally matching compute and filesystem resources with user application needs, for example, judiciously choosing between Lustre, NFS, and local storage options based on detailed profiling of user applications.  Balancing compute resource CPU and memory resources to optimally accommodate primary analysis codes in a cost-efficient manner.
        \item
          Managing system operations with user input and influence; balancing required maintenance activities with user deadlines.
        \item
          Implementing governance model \& processes for dispositoning system reservations and special access requirements.
        \end{itemize}
      \item
        Primary Architect for the CFDLab (UT Austin), Aerolab, and Flight Sciences Labs (NASA/JSC) high-performance computing Linux environments. Led system life-cycle from procurement, through deployment, to operations and retirement for our network, storage, compute, and workstation hardware.
    \begin{itemize}
      \item
        Architected 3 generations of Lustre Filesystems, from 2007 to present.  Selected all server components, MDS/OSS width and depth balance, storage subsystem configuration \& tuning, deployment, operations, and retirement.  Production Lustre experience ranges from v1.6 to v2.14, in a multi-homed InfiniBand/40GbE network environment, including both \texttt{ldiskfs} \& ZFS MDT \& OST technologies, and active/active (MDT) \& active/passive (OST) failover pairing.
      \item
	Architected/selected 7 generations of Linux clusters, from 2000 to present.  Experience dates from very early ‘Beowulf’ 16-node clusters, to (currently) multiple InfiniBand-connected HPE ICE-X/XA systems \& federated under a single SLURM interface with a companion Supermicro blade cluster.
      \item
	Architected a distributed lab environment across the NASA/JSC campus, including network and workstation selection, enabling users’ desktop \& software environment to seamlessly match centralized HPC resources, including mounting the same home and Lustre filesystems.  This provided a significant efficiency for our use cases, where large files are often required for pre- and post-processing of analyses.  Users can develop inputs for PDE analysis directly on the HPC filesystem, and post-process in the same global workspace.  Importantly, data transfer for HPC staging was largely eliminated in this setting.
    \end{itemize}

  \item
    HPC Systems Administrator with primary responsibility for multiple Lustre, ZFS, NFS, and DMF filesystems, multiple Linux Compute clusters, high-performance end-user workstations, and the software environment for 200+ users.
    \begin{itemize}
    \item
      Responsible for maintenance \& operations of Lustre filesystems, including development of custom monitoring tools for use by users and in Ganglia-based diagnostics.  For example, aggregate Lustre Jobstats \& SLURM job information to quickly identify workflows with unusual metadata requirements, bandwidth usage, or both.  Integrated Robinhood-based usage statistics for rapid user filesystem usage information.
    \item
      Developed \& maintained the common, user-facing HPC software environment on multiple Linux systems.  This singly-homed software environment supported 3 clusters and end-user workstations simultaneously, using the \href{https://lmod.readthedocs.io/en/latest}{Lmod Lua module environment} in order to provide transparent support to multiple compiler suites, Python versions, MPI variants (OpenMPI, MPICH, MVAPICH, and HPE/MPT), and requisite development library software stacks simultaneously.
    \item
      Developed utilities for efficiently dispatching and aggregating Monte-Carlo workflows on HPC systems in a Lustre environment, using an extensible Py4MPI interface, performing local I/O to \texttt{tmpfs}-based filesystems, and aggregating large-scale results to a shared Lustre filesystem. The approach allows for separating small file access from the Lustre filesystem wherever practicable, archiving results instead in a small number of large files.
    \end{itemize}
  \item
    HPC Systems User as a finite element \& computational fluid dynamics subject matter expert.
    \begin{itemize}
    \item
      Primary developer of the \href{(http://libmesh.github.io)}{libMesh parallel adaptive framework} (2002-2017), a C++/MPI framework for solving partial differential equations with the finite element method.
    \item
      Hypersonic Aerothermodynamics specialist in support of NASA’s human spaceflight. Developed design databases for numerous spacecraft designs through performing large-scale CFD simulations across a wide range of computing resources and interface, including NASA’s High-end Computing \emph{Columbia} and \emph{Pleiades}, and multiple generations of HPC resources at the Texas Advanced Computing Center (TACC).  Involved the orchestration, execution, and delivery of 1000's of individual HPC simulations at discrete flight configurations.  Required the development of tools to manage multiple batch processes simultaneously in disparate queueing environments at multiple HPC sites.
    \end{itemize}
  \end{itemize}



  \item[12/2003 -- 6/2013]
    Aerospace Engineer, NASA Lyndon B.\ Johnson Space Center, Aeroscience \& Flight Mechanics Division, Applied Aeroscience \& CFD Branch
      %\small
      \begin{itemize}
        \item
          Advanced modeling and simulation technology development for simulation of ablating reentry vehicles.

        \item
          Aerothermodynamic analysis lead in support of Orion Crew Module design.
        \item
          Supported Orion thermal protection system design by developing aerothermodynamic
          design environments and overall system requirements.
        \item
          Designed and executed high-enthalpy wind tunnel and arc-jet tests to provide
          vehicle design information and validation data for computational methods.
        \item
          Served as aerosciences lead for Space Shuttle Orbiter Reinforced Carbon/Carbon (RCC)
          repair activities.
        \item
          Member of the Damage Assessment Team in support of the Space Shuttle Program
          during STS-114--STS-123.  Supported EVAs to execute thermal protection system
          repair during STS-114 and -117.
      \end{itemize}
      %\normalsize

    \item[8/1998 -- 12/2003]
      Research Assistant and System Administrator, The University of Texas at Austin
      %\small
      \begin{itemize}
	\item
	  Parallel algorithm research and development for adaptive,
	  coupled fluid mechanics and heat transfer applications.
	\item
	  Led an international software development effort to produce
          a \href{http://libmesh.sourceforge.net}{finite element framework}
	  for high performance parallel computing platforms.
	\item
	  Research microgravity fluid mechanics using CFD.
	\item
	  Administer a heterogeneous UNIX network.
      \end{itemize}
      %\normalsize

      %% \item[8/1997 -- 5/1999]
      %%   Temporary Professional Co-Op, Lockheed Martin Space Mission Systems \& Services
      %%   \small
      %%   \begin{itemize}
      %%     \item
      %%       Researched feasibility of the human exploration of Mars.
      %%     \item
      %%       Created programs to analyze Space Shuttle plume heating to the
      %%       International Space Station.
      %%     \item
      %%       Conducted multiphase, chemically reacting supersonic flow analysis
      %%       of rocket engines.
      %%     \item
      %%       Upgraded and documented an orbital debris reentry simulation program for
      %%       a major revision release.
      %%     \item
      %%       Generated and analyzed three dimensional grids for hypersonic reentry
      %%       vehicles in chemical and thermal non-equilibrium flow regimes.
      %%   \end{itemize}
      %%   \normalsize
  \end{cvlist}

  \vspace{-.5em}
% Research Interests
  \begin{cvlist}{Research Interests}
    \currentpdfbookmark{Research Interests}{research}
    \item[High-Performance Computing] \mbox{ }
      \begin{itemize}
        %\small
        \item[-] Numerical methods for massively parallel adaptive mesh refinement simulations.
        \item[-] Multilevel parallel algorithm development and implementation.
        \item[-] High-performance computing library design and development.
        \item[-] High-performance computing systems design \& architecture.
      \end{itemize}
    \item[Computational Fluid Dynamics] \mbox{ }
      \begin{itemize}
        %\small
        \item[-] Numerical simulation of hypersonic, thermochemical nonequilibrium flows.
        \item[-] Coupled multiphysics simulation of ablating reentry vehicles.
        \item[-] Stabilized finite element methods for convective-dominated flows.
        \item[-] Numerical methods for biomedical applications.
        \item[-] Fluid-structure interaction.
      \end{itemize}
  \end{cvlist}

% Honors & Awards
  \begin{cvlist}{Honors and Awards}
    \currentpdfbookmark{Honors and Awards}{honors}

    \item[June 2012]
      \textbf{NASA Early Career Achievement Medal.}

      \begin{quote}
        \em For outstanding contributions in support of human spaceflight and foundational discipline advancing efforts in the area of multi-physics modeling and simulation initiatives.
      \end{quote}

    \item[May 2010]
      NASA Group Achievement Award, presented to the {\em Shuttle Boundary Layer Transition Tool Development Team}

    \item[November 2009]
      Exceptional Contribution Award, from the Orion Project Office / Vehicle Integration Office.

    \item[July 2009]
    NASA Group Achievement Award, presented to the {\em CEV Aerosciences Project}.

    \item[May 2009]
      \textbf{Rotary National Award for Space Achievement (RNASA) Foundation {\em Stellar Award} - Early Career Category.}
      \begin{quote}
        \em For outstanding technical contributions in determination of accurate aerothermal environments for safe operation of the Space Shuttle orbiter and development of the Orion spacecraft.
      \end{quote}


    \item[March 2009]
    NASA Group Achievement Award, presented to the {\em Soyuz Assessment Team}.

    \item[March 2008]
    NASA Group Achievement Award, presented to the {\em MH-13 Orbiter Aerothermodynamic Test Team}.

    \item[January 2008]
     RCC Repair Team Award, presented to the Reinforced Carbon/Carbon repair team for designing and implementing an on-orbit repair technique suitable to Orbiter nosecap and wing-leading-edge damage.

    \item[July 2007]
      NASA Lyndon B.~Johnson Space Center Group Achievement Award, presented to the {\em STS-117 OMS Pod Blanket Repair Team}.

    \item[August 2006]
      \textbf{NASA's Astronauts' Personal Achievement {\em Silver Snoopy} Award.}

      \begin{quote}
        \em In recognition of your contributions to the Orbiter Aerothermal Return-to-Flight Team as you took a lead role in the development of a tool used to assess damage to the thermal protection system.
      \end{quote}

    \item[April 2006]
      NASA Group Achievement Award, presented to the {\em Orbiter Aerothermodynamics Working Group}.

    \item[January 2001]
      US Department of Energy Computational Science Graduate Fellowship.

    \item[August 2000]
      Texas Institute of Computational and Applied Mathematics (TICAM) Fellowship.

    \item[August 1999]
      John C.\ Westkaemper Award for academic excellence.
%%     \item[December 1997]
%%       Lockheed Martin Lighting Award for exceptional performance.
  \end{cvlist}


% Fellowships and Assistantships
  \begin{cvlist}{Fellowships and Assistantships}
    \currentpdfbookmark{Fellowships and Assistantships}{fellow}
    \item[1/2001 -- 12/2003]
      U.S. Department of Energy Computational Science Graduate Fellowship
    \item[8/2000 -- 12/2002]
      Graduate Fellowship,
      Institute for Computational and Engineering Sciences (ICES),
      The University of Texas at Austin
    \item[8/2000 -- 12/2003]
      Graduate Research Assistantship,
      Aerospace Engineering Department,
      The University of Texas at Austin.
    \item[8/1998 -- 5/2000]
      Undergraduate Research Assistantship,
      Aerospace Engineering Department
      The University of Texas at Austin.
  \end{cvlist}


% Publications Section
  \begin{cvlist}{Publications}
    \currentpdfbookmark{Publications}{publications}

    % Journal Articles
    \subpdfbookmark{Journal Articles}{journal}
    \bibliographystylejournal{ieeetr}
    \nocitejournal{fins_cnf14,                                           %% 2014
                   adiabaticSC10,fins_aiaa10,                            %% 2010
                   kirk_chemotaxis,wterror09,                            %% 2009
                   ctwatch08,                                            %% 2008
                   fins_ijnmf, %benkirk_dissertation,                    %% 2007
                   libMeshPaper,jsr06,tmj06,                             %% 2006
                   modelling_error,carey_bail_2004,                      %% 2004
                   kirk_lipnikov_cmg}                                    %% 2003
    \item[\underline{Journal Articles}]    \bibliographyjournal{std}

    % Conference Proceedings Articles
    \currentpdfbookmark{Conference Proceedings}{proc}
    \bibliographystyleproc{ieeetr}
    \nociteproc{aiaa-2013-2559,aiaa-2013-306,aiaa-2013-33,                 %% 2013
                aiaa-2011-3779,aiaa-2011-3841,aiaa-2011-144,aiaa-2011-134, %% 2011
                aiaa-2010-1560,aiaa-2010-1183,aiaa-2010-1462,              %% 2010
                aiaa-2009-730,                                             %% 2009
                aiaa-2008-1226,aiaa-2008-921,                              %% 2008
                aiaa-2007-603,                                             %% 2007
                aiaa-2006-2922,aiaa-2006-2921,                             %% 2006
                cfd05,                                                     %% 2005
                afc01,                                                     %% 2001
                world_scientific,hpc00}                                    %% 2000
    \item[\underline{In Proceedings}]    \bibliographyproc{std}

%     % Presentations
%     \currentpdfbookmark{Presentations}{pres}
%     \bibliographystyleconf{ieeetr}
%     \nociteconf{pecos_08,arl_instability_08,arl_multicore_08,                      %% 2008
%                 sandia07,defense07,urgent07,fef07,rodeo07,acdl07,                  %% 2007
%                 cse05_1,cse05_2,                                                   %% 2005
%                 siam04,                                                            %% 2004
%                 barth03cluster, barth03message,                                    %% 2003
%                 geosciences03,
%                 carey02advances,                                                   %% 2002
%                 barth01newtonian,carey01parallel,carey01recent,barth01beowulf,     %% 2001
%                 barth01parallel,cm01,amflow01,
%                 carey00mgflo,carey00distributed,carey00parallel_2,carey00parallel, %% 2000
%                 barth98beowulf}                                                    %% 1998
%     \item[Presentations]    \bibliographyconf{std}

% %     % Technical Reports
% %     \currentpdfbookmark{Techical Reports}{tr}
% %     \bibliographystylerep{ieeetr}
% %     \nociterep{aeroheating98,orsat99}
% %     \item[Technical Reports]    \bibliographyrep{std}

    \end{cvlist}

  % Teaching section
  \begin{cvlist}{Teaching}
    \pdfbookmark[0]{Teaching}{teaching}

    % Short Courses
    \subpdfbookmark{Short Courses}{shortcourses}
    \bibliographystyleconf{ieeetr}
    \nociteshortcourses{hypersonics21,                                        %% 2021
                        prace13,                                              %% 2013
                        arl08,                                                %% 2008
                        erdc07,                                               %% 2007
                        fe_jsc04,wpafb04,                                     %% 2004
                        arl03,uwa03}                                          %% 2003
    \item[\underline{Short Courses}]    \bibliographyshortcourses{std}

    % Semester Courses
    \vspace{1em}
    \currentpdfbookmark{Full Courses}{fullcourses}
    \item[\underline{Full Courses}]
      \begin{enumerate}
        \item ASE 376K, \emph{Propulsion}, The University of Texas at Austin, Fall 2009.
        \item EM 393N, \emph{Numerical Methods in Flow and Transport}, The University of Texas at Austin, Spring 2010. (co-taught with R.~Stogner and P.~Bauman.)
        \item MECH 691 \emph{Hypersonic Aerodynamics}, Rice University, Fall 2011.
      \end{enumerate}
  \end{cvlist}


  % Course List
  \begin{cvlist}{Graduate Course History}
    \pdfbookmark[0]{Graduate Course History}{courses}
      \item[Fluid Mechanics]

	\begin{enumerate}
  	  \item \emph{Foundations of Fluid Mechanics}, Fall 2000.
	  \item \emph{Viscous Fluid Flow}, Fall 2001.
	  \item \emph{Compressible Fluid Mechanics}, Spring 2001.
	  \item \emph{Hypersonic Aerodynamics}, Fall 2001.
	  \item \emph{Molecular Gas Dynamics}, Spring 2002.
	  \item \emph{Advanced Problems in Compressible Flow}, Fall 2002.
	\end{enumerate}

      \vspace{.5em}
      \item[Numer.\ Meth.]
	\begin{enumerate}
	  \item \emph{Finite Element Methods}, Fall 2000.
	  \item \emph{Numerical Methods for Flow \& Transport Problems}, Fall 2000.
	  \item \emph{Advanced Methods in Computational Mechanics}, Spring 2001.
	  \item \emph{Grid Generation \& Adaptive Grids}, Fall 2001.
	  \item \emph{Numerical Analysis: Numerical Linear Algebra}, Fall 2001.
	  \item \emph{Numerical Simulation of Transport in Semiconductors}, Summer 2002.
	  \item \emph{Lagrangian Methods in Computational Fluid Dynamics}, Fall 2002.
	  \item \emph{Advanced Computational Flows \& Transport}, Fall 2002.
	\end{enumerate}

      \vspace{.5em}
      \item[Mathematics]

	\begin{enumerate}
  	  \item \emph{Matrices and Matrix Calculations}, Summer 2000.
	  \item \emph{Mathematical Methods in Applied Mechanics}, Fall 2000.
	  \item \emph{Mathematical Methods in Applied Mechanics II}, Spring 2001.
	\end{enumerate}

      \vspace{.5em}
      \item[Comp.\ Sci.]

	\begin{enumerate}
	  \item \emph{High-Performance Graphics \& Visualization}, Spring 2002.
	  \item \emph{High-Performance \& Parallel Computing}, Spring 2002.
	\end{enumerate}
  \end{cvlist}

% Skills
  \begin{cvlist}{Tools \& Skills}
    \item[]
    Lustre architecting \& operations, HPC Linux Administration \& Power User Practitioner. SLURM configuration \& batch queuing operations. Message-passing \& thread-based parallelism. MPI application development \& optimization. NFS, ZFS, \& High-availability technologies. CentOS deployment and operations (bare metal and virtualized). HPE/SGI ICE-X/XA cluster deployment and operations. HPC Applications development, support, porting, optimization, and debugging. Fluent in bash, C++, Python, \LaTeX{}, GNU Autotools, git/svn, emacs, and English. Applied Mathematics. Finite Element Analysis. Technical leadership and communication. Personnel Management. Aerodynamics, Aerothermodynamics, Parachute/Decelerator Systems. Spacecraft Design, Development, Test, \& Engineering.
  \end{cvlist}



{\hfill \hrule\hfill\vspace{.1mm} \\}
\end{cv}

\end{document}

% LocalWords:  hypersonic Aeroscience HPC Architected Lustre MDS OSS
% LocalWords:  Filesystems InfiniBand GbE ldiskfs ZFS OST failover XA
% LocalWords:  HPE SLURM Supermicro JSC filesystems pre PDE NFS DMF
% LocalWords:  filesystem Jobstats Robinhood Lmod Lua MPI OpenMPI MPT
% LocalWords:  MPICH MVAPICH Py tmpfs libMesh Aerothermodynamics CFD
% LocalWords:  TACC queueing Aerosciences Aerosicence SpaceX
% LocalWords:  aerothermodynamic
